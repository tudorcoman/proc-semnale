\begin{abstractpage}

\begin{abstract}{romanian}
    
    Această lucrare explorează tehnici pentru detectarea anomaliilor în tranzacțiile bursiere și analizează impactul acestora asupra eficienței piețelor financiare. Scopul principal al studiului este identificarea și caracterizarea evenimentelor neobișnuite care pot afecta performanța activelor financiare. Pentru a valida eficacitatea metodelor propuse, se folosesc seturi de date reale provenind de pe diverse piețe financiare. Sunt prezentate atât metode din analiza semnalelor, cât şi din analiza statistică.

\end{abstract}

\end{abstractpage}